\begingroup
\renewcommand{\cleardoublepage}{}
\clearpage
\extra{Préambule}
\endgroup

\extras{Contexte}
Depuis des décennies l'utilisation des cartes de crédit pour réaliser de achats à augmenter de
manière exponentielle. Le premier facteur fut la part de marché de plus en plus grande pour
les achats en ligne. Ces derniers temps, ce fut une nouvelle accélération pour cause de
pandémie. Les choses ont été rapides même trop rapide. \nextpar
Vous travaillez au sein d'une société spécialisée dans la sécurité des systèmes bancaires.
Vous disposez d'un simulateur de paiement Mobile Money.
L'étude de cas de simulation de paiement Mobile Money est basé sur une entreprise réelle
qui a développé une implémentation d'argent mobile qui offre aux utilisateurs de téléphones
mobiles la possibilité de transférer de l'argent entre eux en utilisant leur téléphone comme un
porte-monnaie électronique. \nextpar
Une nouvelle vient d'arriver et les données sont dorénavant suffisantes pour mettre en place
un système de prédiction des fraudes à la carte bancaire.
La tâche à accomplir est de développer une approche qui détecte les activités suspectes qui
sont révélatrices de fraude.

\extras{Planification}
    Pour la partie planification, nous avons décidé d'utiliser \textbf{Notion}, une application Web freemium de productivité et de prise de notes qui propose des outils d'organisation, notamment la gestion des tâches, le suivi des projets, les listes de tâches et la mise en favoris.
    Le tableau qui suit décrit notre organisation.
    \begin{arabic-enum}
        \item Planification de la réunion de lancement;
        \item Rédaction d'une proposition de projet;
        \item Recontextualisation du projet : \emph{A partir de notre base de données (fichier \ilc{credit\_card\_fraud.csv}), développer un modèle de prédiction qui permet determiner si une transaction est une fraude ou non};
        \item Réalisation d'un MCD/MPD;
        \item Construction de la base de données;
        \item Dataviz
        \item Choix, construction et évaluation de quatres modèles de classification : \ilc{XGBoost}, \ilc{AdaBoost}, \ilc{KNN}, \ilc{RandomForest};
        \item Construction et réalisation de la maquette.
    \end{arabic-enum}
